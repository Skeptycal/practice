\documentclass[dvipdfmx,cjk]{beamer} 
\usepackage{graphicx}
\usepackage{tikz}
\usetheme{CambridgeUS} 

\begin{document}


\title[Halide]{ステンシル計算言語Halide}
\author[T. Muranushi]{村主崇行}            %% ここに書かれた情報は色々なところに使われるので
\institute[RIKEN/AICS]{計算科学研究機構}   %% なるべく設定した方が良い

\begin{frame}                  %% \begin{frame}..\end{frame} で 1 枚のスライド
\titlepage                     %% タイトルページ
\end{frame}

\begin{frame}                  %% 目次 (必要なければ省略)
\tableofcontents
\end{frame}

\section{はじめに} 
\begin{frame}\frametitle{Halideとは?}
Halideはステンシル計算プログラムの生成とチューニングのためのライブラリ。
画像処理のために開発された。\cite{ragan2012decoupling,ragan2013halide}


\begin{center}
  \begin{tabular}{|c|c|c|}
    \hline
    &Paraiso & Halide\\
    \hline
    基本言語 & Haskell & C++ \\
    コード生成対象 & x86, CUDA &
    \multicolumn{1}{p{5cm}|}{x86/SSE, ARM, Native Client, OpenCL, CUDA,  ... }\\
    扱える次元 & n次元 & n次元 \\ 
    最適化の種類 & ループ融合、同期 & \multicolumn{1}{p{5cm}|}{並列度、局所性、計算節約 の間のトレードオフ} \\ 
    自動チューニング & あり & なし \\ 
    \hline
  \end{tabular}
\end{center}
\end{frame}



\begin{frame}[allowframebreaks]{参考文献}{}
  \bibliographystyle{apalike}
  \bibliography{bunken}
\end{frame}



% \begin{bibliography}
% 
% \bibitem[de~Saussure, 1995]{Saussure1995}
% de~Saussure, F. (1995).
% \newblock {\em Cours de Linguistique Grale}.
% \newblock Payot.
% 
% \bibitem[Labov, 1972]{Labov1972}
% Labov, W. (1972).
% \newblock {\em Sociolinguistic Patterns}.
% \newblock University of Pennsylvania Press, Philadelphia.
% 
% \end{bibliography}








\section*{リサイクルボックス}




\subsection{}
\begin{frame}\frametitle{}
\begin{eqnarray}
\end{eqnarray}
\end{frame}


\begin{frame}[fragile]\frametitle{}
\begingroup
    \fontsize{8pt}{9pt}\selectfont
\begin{verbatim}
subroutine kernel__velderiv()
\end{verbatim}
\endgroup
\vspace{-1cm}
\end{frame}




\end{document}

