%\documentclass[dvipdfmx,cjk]{beamer}   %% よって変える
\documentclass{beamer}  %% オプションは環境や利用するプログラムに
\usepackage{luatexja}
\usepackage{tikz}
\usepackage{color}
\usepackage{natbib}
%\usepackage{bibentry}
% \bibliographystyle{abbrvnat}
\usepackage{chngcntr}
\usepackage{amsmath}
\usepackage{graphicx}
\usetikzlibrary{positioning}
\usetikzlibrary{calc}



\usepackage{perpage}
% use symbolic footnotes, instead of numbers
\renewcommand*{\thefootnote}{\fnsymbol{footnote}}
\MakePerPage{footnote}

%\AtBeginDvi{\special{pdf:tounicode 90ms-RKSJ-UCS2}} %% しおりが文字化けしないように
\AtBeginDvi{\special{pdf:tounicode EUC-UCS2}}
\AtBeginSection{\frame{\sectionpage}}


%ページ内で引用を済ます
\counterwithin*{footnote}{page}
\newcommand\footcite[1]{\footnote{\tiny \bibentry{#1}}\label{\thepage:#1}}
\newcommand\secondcite[1]{\textsuperscript{\ref{\thepage:#1}}}


%\setbeamertemplate{navigation symbols}{} %% 右下のアイコンを消す

\usetheme{CambridgeUS}         %% theme の選択
%\usetheme{Boadilla}           %% Beamer のディレクトリの中の
%\usetheme{Madrid}             %% beamerthemeCambridgeUS.sty を指定
%\usetheme{Antibes}            %% 色々と試してみるといいだろう
%\usetheme{Montpellier}        %% サンプルが beamer\doc に色々とある。
%\usetheme{Berkeley}
%\usetheme{Goettingen}
%\usetheme{Singapore}
%\usetheme{Szeged}

%\usecolortheme{rose}          %% colortheme を選ぶと色使いが変わる
%\usecolortheme{albatross}

%\useoutertheme{shadow}                 %% 箱に影をつける
%\usefonttheme{professionalfonts}       %% 数式の文字を通常の LaTeX と同じにする

%\setbeamercovered{transparent}         %% 消えている文字をうっすらと表示する

\setbeamertemplate{theorems}[numbered]  %% 定理に番号をつける
\newtheorem{thm}{Theorem}[section]
\newtheorem{proposition}[thm]{Proposition}
\theoremstyle{example}
\newtheorem{exam}[thm]{Example}
\newtheorem{remark}[thm]{Remark}
\newtheorem{question}[thm]{Question}
\newtheorem{prob}[thm]{Problem}

\newcommand\done[1]{\textcolor{blue}{#1 済}}

\begin{document}

\bibliographystyle{apalike}
\nobibliography{bunken}

\title[Halide]{ステンシル計算言語Halide}
\author[T. Muranushi]{村主崇行}            %% ここに書かれた情報は色々なところに使われるので
\institute[RIKEN/AICS]{計算科学研究機構}   %% なるべく設定した方が良い
%\date{April 23, 2014}

\begin{frame}                  %% \begin{frame}..\end{frame} で 1 枚のスライド
\titlepage                     %% タイトルページ
\end{frame}

\begin{frame}                  %% 目次 (必要なければ省略)
\tableofcontents
\end{frame}

\section{はじめに} 
\begin{frame}\frametitle{Halideとは?}
\citet{johnson2008notes}
ステンシル計算プログラムの生成とチューニングのためのライブラリ。
画像処理のために開発された。\citep{moczo2007finite}


\begin{center}
  \begin{tabular}{|c|c|c|}
    \hline
    &Paraiso & Halide\\
    \hline
    基本言語 & Haskell & C++ \\
    コード生成対象 & x86, CUDA &
    \multicolumn{1}{p{5cm}|}{x86/SSE, ARM v7/NEON, CUDA, Native Client, OpenCL }\\
    扱える次元 & n次元 & n次元 \\ 
    最適化の種類 & ループ融合、同期 & xxx \\ 
    自動チューニング & あり & なし \\ 
    \hline
  \end{tabular}
\end{center}
%\citet{moczo2007finite}
\end{frame}





\begin{frame}[allowframebreaks]{参考文献}{}

\end{frame}




% \begin{bibliography}
% 
% \bibitem[de~Saussure, 1995]{Saussure1995}
% de~Saussure, F. (1995).
% \newblock {\em Cours de Linguistique Grale}.
% \newblock Payot.
% 
% \bibitem[Labov, 1972]{Labov1972}
% Labov, W. (1972).
% \newblock {\em Sociolinguistic Patterns}.
% \newblock University of Pennsylvania Press, Philadelphia.
% 
% \end{bibliography}








\section*{リサイクルボックス}




\subsection{}
\begin{frame}\frametitle{}
\begin{eqnarray}
\end{eqnarray}
\end{frame}


\begin{frame}[fragile]\frametitle{}
\begingroup
    \fontsize{8pt}{9pt}\selectfont
\begin{verbatim}
subroutine kernel__velderiv()
\end{verbatim}
\endgroup
\vspace{-1cm}
\end{frame}




\end{document}

