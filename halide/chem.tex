%\documentclass{beamer}  %% オプションは環境や利用するプログラムに
%\usepackage{luatexja}
%\usetheme{CambridgeUS} 


\documentclass[dvipdfmx,cjk]{beamer} 
\usepackage{graphicx}
\usepackage{tikz}
\usetheme{CambridgeUS} 


\begin{document}
\section{How 2 BibTeX on Beamer}
\begin{frame} \frametitle{My theory in one slide}
水は酸素と水素からできているのです \cite{moczo2007finite}.
  \begin{eqnarray}
    2H_2 + O_2 \to 2H_2O
  \end{eqnarray}
\end{frame}

\begin{frame}[allowframebreaks]{参考文献}{}
  \bibliographystyle{apalike}
\begin{thebibliography}{}

\bibitem[Johnson, 2008]{johnson2008notes}
Johnson, S.~G. (2008).
\newblock Notes on perfectly matched layers (pmls).
\newblock {\em Lecture notes, Massachusetts Institute of Technology,
  Massachusetts}.

\bibitem[Moczo et~al., 2007]{moczo2007finite}
Moczo, P., Robertsson, J.~O., and Eisner, L. (2007).
\newblock The finite-difference time-domain method for modeling of seismic wave
  propagation.
\newblock {\em Advances in Geophysics}, 48:421--516.

\bibitem[Moczo et~al., 2008]{moczo2008finite}
Moczo, P., Robertsson, J.~O., and Eisner, L. (2007).
\newblock The finite-difference time-domain method for modeling of seismic wave
  propagation.
\newblock {\em Advances in Geophysics}, 48:421--516.

\bibitem[Moczo et~al., 2009]{moczo2009finite}
Moczo, P., Robertsson, J.~O., and Eisner, L. (2007).
\newblock The finite-difference time-domain method for modeling of seismic wave
  propagation.
\newblock {\em Advances in Geophysics}, 48:421--516.

\bibitem[Moczo et~al., 2010]{moczo2010finite}
Moczo, P., Robertsson, J.~O., and Eisner, L. (2007).
\newblock The finite-difference time-domain method for modeling of seismic wave
  propagation.
\newblock {\em Advances in Geophysics}, 48:421--516.

\end{thebibliography}
\end{frame}

\end{document}
