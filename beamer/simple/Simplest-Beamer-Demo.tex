\documentclass[t,cjk]{beamer}
\usepackage{luatexja}
\usepackage{amsmath}
\usepackage{natbib}
\def\newblock{} %% For beamer and natbib to coexist
\setbeamertemplate{navigation symbols}{} % remove beamer navigation symbols
\usetheme{CambridgeUS}         %% theme
\usefonttheme{professionalfonts}       %% 数式の文字を通常の LaTeX と同じにする
\setbeamercovered{transparent}         %% 消えている文字をうっすらと表示する
\usepackage{concmath} %% プロジェクターでも読みやすい太めのフォント。 texlive-math-extraに入っている


\bibliographystyle{apj}


\begin{document}
\title[Short title]{Solar Flare Forecast by Deep Learning}
\author[T. Muranushi]{Takayuki Muranushi}            %% ここに書かれた情報は色々なところに使われるので
\institute[RIKEN AICS]{RIKEN Advanced Institute for Computational Science}
\date{Nov 25, 2015}

\begin{frame}                  %% \begin{frame}..\end{frame} で 1 枚のスライド
  \titlepage                     %% タイトルページ
  \begin{center}
    for seminar/meeting
  \end{center}
\end{frame}

\begin{frame}                  %% 目次 (必要なければ省略)
\tableofcontents
\end{frame}

\section{Our mission in space weather forecast}


\frame{
  \frametitle{Accepted for publication in Space Weather!}
}




\frame{  \frametitle{Introduction}
  \begin{align}
    a &=b+c_d\phi
  \end{align}

  Space weather is important, and space weather forcast using
  statistical methods has been practiced e.g. by
  \citet{song2009statistical},
  \citet{bloomfield2012toward},\citet{2013SoPh..283..157A},
  \citet{bobra2014solar}, and \citet{muranushi2015ufcorin}

}



\appendix

\section{Bibliography}
\frame[allowframebreaks]{
  \frametitle{Bibliography}
  \bibliography{reference.bib}
}

\end{document}
